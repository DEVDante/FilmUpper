\documentclass[twoside]{projektInzynierskiMS}
\usepackage{polski}
\usepackage{lmodern}
\usepackage[utf8]{inputenc}
\usepackage{amsmath}
%\drukJednostronny

%% tytuł promotor iautor (\title to komenda standardowa)
\title{FilmUpper}
\promotor{dr inż. Adam Zielonka}


%% każdy autor musi mieć 4 argumenty: imię nazwisko, nr albumu, procent wkładu, opis wkładu
\autor{Kamil Rutkowski}{112233}{50} {Struktura aplikacji, Algorytmika}
	
\autor{Malwina Borecka-Xsinska}{112233}{50}	{Kodowanie i dekodowanie plików, Interfejs użytkownika, Algorytmika}

	
	


%% dedykacja mile widziana
\dedykacja{To jest\\dedykacja}
%\NumeryNaPoczatku
%% numeracja wzorów tu włączona typu (1.2.3), ta druga to typu (1.2), domyślnie typu (1)
%\subsectionWzory
% \sectionWzory  

%\rozdzialy


%\literowaNumeracjaDodatkow %% włączy numerację dodatków literami
%\rzymskaNumeracjaDodatkow  %%włączy numerację dodatków liczbami rzymskimi

%% wyłączenie wyjaśnień:
\bezWyjasnien

%% standardowe komendy \newtheorem  działają jak woryginale
\newtheorem{tw}{Twierdzenie}%[subsection]
\newtheorem{twa}{Twierdzenie}%[section]
\newtheorem{dd}{Definicja}%[subsection]

\begin{document}


FilmUpper jest aplikacją pozwalającą na poprawianie jakości obrazu w filmie, poprzez zwiększanie rozdzielczości oraz zwiększanie ilości klatek na sekundę w nim występójących. Dzięki takim zabiegom jakość oglądanego przez nas obrazu znacząco się poprawia, jednakże nigdy nie będzie ona tak dobra, jak jakość obrazu nagrywanego z ustawieniami na które chcemy dany film przekonwertować. 



\section{Technologia}


Przetwarzanie plików filmowych nawet w trywialny sposób jest zadaniem bardzo wymagającym wydajnościowo, dlatego wybór technologii miał kluczowe znaczenie dla wydajności całej aplikacji. Musieliśmy rozwarzyć wiele czynników które mogłyby wpłynąć na działanie programu oraz na sposób napisania go.


\subsection{Język programowania}

Wybór języka programowania był dla nas kluczowy, ze względu na to, że znacząco wpływa na prędkość działania programu, dostępne narzędzia i biblioteki. Znajomość danego języka także była dla nas jednym z kluczowych czynników przy jego wyborze. Naszym rozwarzaniom poddaliśmy następujące języki programowania.

\subsubsection{C++}
Język C++ jest jednym z najczęściej używanych języków niskopoziomowych. Jego popularność jest skutkiem bardzo długiego czasu na rynku oraz pewnej prostoty użycia. Kolejne wersje tego wciąż rozwijanego języka dodają nowe, sprawdzone i ułatwiające tworzenie programów rozwiązania z innych języków programowania. Bardzo duża wydajność oraz mnogość dostępnych bibliotek związanych z dekodowaniem i enkodowaniem plików filmowych jest bardzo ważnym aspektem tego wyboru. Wybór ten wiązałby się także z kilkoma negatywnymi cechami tego języka, wymóg ręcznego zarządzania pamięcią, mało przejrzysta składnia przy tworzeniu rozwiązań o dużym stopniu skomplikowania oraz brak ułatwień które poznaliśmy w językach wysokopoziomowych są jednymi z nich. Biorąc pod uwagę naszą stosunkowo długą pracę z tym językiem oraz wszystkie za i przeciw oceniliśmy, że będzie on najlepszym wyborem.

\subsubsection{C\#}
Wysokopoziomowe rozwinięcie języka z rodziny C. Mimo posiadania składni podobnej do C++, język ten porzuca wiele z nieprzyjemnych jego aspektów. Poprzez automatyczne zarządzanie pamięcią, usunięcię składni charakterystycznej dla wskaźników oraz dodanie wielu nowych mechanizmów, wygląd kodu oraz prędkość tworzenia programów znacząco wzrasta. Bardzo ważnym składnikiem C\# jest także LINQ które umożliwia bardzo kompaktowe i przejrzyste działanie na kolekcjach co jest dużym plusem przy przetwarzaniu plików wideo, jako że są one kolekcjami pojedyńczych klatek złożonych z kolekcji pikseli. Wszystkie z tych udogodnień mają jednak cenę w postaci mniejszej wydajności w stosunku do C++ oraz brak wystarczającego wsparcia dla zarządzania plikami wideo jako że język ten jest stworzony z myślą o szybkim tworzeniu aplikacji biurowych. Niestety, brak bibliotek dedykowanych obróbce plików wideo w wymaganym przez nas stopniu był głównym powodem, dla którego nie wybraliśmy tego języka.

\subsubsection{Rust}
Prędkość działania porównywalna z językiem C++, duże bezpieczeństwo pod względem zarządzania pamięcia, łatwość w konwersji programu jednowątkowego na wielowątkowy, składnia języka oraz wiele udogodnień zaciągniętych z języków wysokiego poziomu. To jedne z wielu punktów które zachęcały do wyboru tego języka. Niestety, brak lub wczesna wersja bibliotek które byłyby niezbędne przy tym projekcie oraz niewielkie doświadczenie z tym językiem sprawiły, że musieliśmy porzucić pomysł użycia go.


\subsection{Dekodowanie oraz enkodowanie filmów}
\subsection{Interfejs użytkownika}


%% UWaga na \newlineTekst oraz \newlineSpis. Można też użyć \newline, działa jak %%\newlineSpis\newlineTekst
\section{Struktura programu}

\section{Algorytmy}


%\dodatek{Mój specjalny dodatek}

%Tu treść dodatku. Zwróćmy uwagę na sposób numerowania dodatku, 
%możliwa jest zmiana numerowania, patrz wyjaśnienia.
          
%% to wpisuje się do spisu treści, ale bez numeru rozdziału,
%% można też używać \dodatek{Tytuł}, który jest numerowany, ale inaczej niż rozdziały.
\dodatkowo{Rysunki}

Tu rysunki

\dodatkowo{Programy}

Tu programy

\begin{verbatim}
#include <stdio.h>

int main()
{
   printf("Hello world\n");
}
\end{verbatim}

\noindent
Oraz 

\bigskip

\vrule\hspace{10pt}\begin{minipage}{10cm}
\begin{verbatim*}
<?php
   echo "test=$test";
?>
\end{verbatim*}
\end{minipage}

\begin{tw}
Twierdzenie Twierdzenie Twierdzenie Twierdzenie Twierdzenie 
\end{tw}
\begin{thebibliography}{12}

\bibitem{PozNazwa1} Jakaś pozycja literatury
\bibitem{InnPoz} Jakaś pozycja literatury

\end{thebibliography}
\end{document}
